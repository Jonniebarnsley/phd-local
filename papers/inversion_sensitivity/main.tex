%% Copernicus Publications Manuscript Preparation Template for LaTeX Submissions
%% ---------------------------------
%% This template should be used for copernicus.cls
%% The class file and some style files are bundled in the Copernicus Latex Package, which can be downloaded from the different journal webpages.
%% For further assistance please contact Copernicus Publications at: production@copernicus.org
%% https://publications.copernicus.org/for_authors/manuscript_preparation.html

% Force LaTeX to look in current directory first
\makeatletter
\def\input@path{{./}}
\makeatother

%% Please use the following documentclass and journal abbreviations for preprints and final revised papers.

%% 2-column papers and preprints
\documentclass[journal abbreviation, manuscript]{copernicus}


%% Journal abbreviations (please use the same for preprints and final revised papers)


% Advances in Geosciences (adgeo)
% Advances in Radio Science (ars)
% Advances in Science and Research (asr)
% Advances in Statistical Climatology, Meteorology and Oceanography (ascmo)
% Aerosol Research (ar)
% Annales Geophysicae (angeo)
% Archives Animal Breeding (aab)
% Atmospheric Chemistry and Physics (acp)
% Atmospheric Measurement Techniques (amt)
% Biogeosciences (bg)
% Climate of the Past (cp)
% DEUQUA Special Publications (deuquasp)
% Earth Surface Dynamics (esurf)
% Earth System Dynamics (esd)
% Earth System Science Data (essd)
% E&G Quaternary Science Journal (egqsj)
% EGUsphere (egusphere) | This is only for EGUsphere preprints submitted without relation to an EGU journal.
% European Journal of Mineralogy (ejm)
% Geochronology (gchron)
% Geographica Helvetica (gh)
% Geoscience Communication (gc)
% Geoscientific Instrumentation, Methods and Data Systems (gi)
% Geoscientific Model Development (gmd)
% History of Geo- and Space Sciences (hgss)
% Hydrology and Earth System Sciences (hess)
% Journal of Bone and Joint Infection (jbji)
% Journal of Environmentally Compatible Air Transport System (jecats)
% Journal of Micropalaeontology (jm)
% Journal of Sensors and Sensor Systems (jsss)
% Magnetic Resonance (mr)
% Mechanical Sciences (ms)
% Natural Hazards and Earth System Sciences (nhess)
% Nonlinear Processes in Geophysics (npg)
% Ocean Science (os)
% Polarforschung - Journal of the German Society for Polar Research (polf)
% Proceedings of the International Association of Hydrological Sciences (piahs)
% Proceedings of the International Ocean Drilling Programme (piodp)
% Safety of Nuclear Waste Disposal (sand)
% Scientific Drilling (sd)
% SOIL (soil)
% Solid Earth (se)
% State of the Planet (sp)
% The Cryosphere (tc)
% Weather and Climate Dynamics (wcd)
% Web Ecology (we)
% Wind Energy Science (wes)


%% \usepackage commands included in the copernicus.cls:
%\usepackage[german, english]{babel}
%\usepackage{tabularx}
%\usepackage{cancel}
%\usepackage{multirow}
%\usepackage{supertabular}
%\usepackage{algorithmic}
%\usepackage{algorithm}
%\usepackage{amsthm}
%\usepackage{float}
%\usepackage{subfig}
%\usepackage{rotating}


\begin{document}

\title{Sensitivity of Antarctic sea level contribution to bed friction inversion choices in the BISICLES ice sheet model}


% \Author[affil]{given_name}{surname}

\Author[1][jonathan.barnsley@kcl.ac.uk]{Jonathan R}{Barnsley} %% correspondence author
\Author[1]{Tamsin L}{Edwards}
\Author[1]{Alex}{Bradley}
\Author[2]{Stephen L}{Cornford}
\Author[2]{Matt}{Trevers}

\affil[1]{King's College London, Department of Geography, London, UK}
\affil[2]{University of Bristol, Department of Geography, Bristol, UK}

%% The [] brackets identify the author with the corresponding affiliation. 1, 2, 3, etc. should be inserted.

%% If an author is deceased, please add \deceased[$Deceased date if applicable$]{$Author number$} (e.g. \deceased[13 November 2015]{2}) at the end of the affiliations. The author number depends on the placement of the author in the author list, e.g. the third author has number 3.


%% If authors contributed equally, please add \equalcontrib{$Author numbers$} (e.g. \equalcontrib{1,3}) at the end of the affiliations. The author number depends on the placement of the author in the author list, e.g. the third author has number 3.




\runningtitle{Sensitivity of Antarctic sea level contribution to bed friction inversion choices in the BISICLES ice sheet model}

\runningauthor{Barnsley et al.}





\received{}
\pubdiscuss{} %% only important for two-stage journals
\revised{}
\accepted{}
\published{}

%% These dates will be inserted by Copernicus Publications during the typesetting process.


\firstpage{1}

\maketitle



\begin{abstract}
Antarctica is the largest potential contributor to future sea level but is also the most uncertain. The ice sheet initial state is known to be a significant source of uncertainty, but this uncertainty is rarely explored outside of large-scale model intercomparison projects.

Century-scale projections of the Antarctic ice sheet typically generate an initial state by solving an inverse problem for basal friction – and sometimes other fields – such that modelled ice velocities closely match observations. While this yields realistic short-term behaviour, it may also overfit by masking deficiencies elsewhere in the model.

Few modelling studies explicitly treat the inverse problem itself as a source of uncertainty. There is some evidence that regularisation choices within the inversion can produce uncertainty comparable to other commonly perturbed ice sheet model parameters, though this is limited to select experimental setups. It is unknown whether these results translate to other ice sheet models or domains.

Here, we use the BISICLES ice sheet model to explore inverse problems as a source of uncertainty in Antarctica’s future sea level contribution. We generate an ensemble of initial states and project them forward to 2300 under a high-emission scenario. We assess the sensitivity of sea level projections to regularisation parameters within the inversion and compare it against other commonly perturbed parameters. We find that the sensitivity to regularisation in BISICLES is generally lower than other parameters and that this gap widens on longer timescales.
\end{abstract}


%\copyrightstatement{} %% This section is optional and can be used for copyright transfers.


\introduction  %% \introduction[modified heading if necessary]



\section{Methods}

\subsection{Ice sheet model}
\subsection{Inversion method}
\subsection{Experimental design}
\subsection{Climate forcing}

\section{Results}

We project our initial states forward with the CESM2 climate forcing up to 2300 and calculate total Antarctic sea level contribution for each ensemble member (Figure~\ref{fig:ensemble_timeseries}). The ensemble range at 2100 is ??--?? m, which compares well with the IPCC AR6 range of 0.03--0.28 m under SSP5--8.5 (cite IPCC). In the 22nd and 23rd centiries, all ensemble members accelerate their mass loss and the range widens to ??--?? m by 2300.

\begin{figure}[t]
\centering
\includegraphics[width=\linewidth]{figures/ensemble.png}
\caption{Timeseries of Antarctic sea level contribution for high and low values of each perturbed parameter.}\label{fig:ensemble_timeseries}
\end{figure}

The spatial patterns of ice thickness change are broadly similar across the ensemble, with most mass loss occuring in regions of West Antarctica: the Amundsen Sea sector, the Siple coast, and the Weddell sea sector. However, the extent of mass loss in these regions depends on parameter choices, with high variance along Thwaites, Foundations, and Denman glaciers (Figure~\ref{fig:spatial}). Some areas of low variance also show regions that retreat in all ensemble members, including the collapse of Pine Island Glacier and the Ross and Filchner-Ronne ice shelves.

\begin{figure}[t]
\centering
\includegraphics[width=\linewidth]{figures/spatial.png}
\caption{Timeseries of Antarctic sea level contribution for high and low values of each perturbed parameter.}\label{fig:spatial}
\end{figure}

For each parameter, we calculate the range in sea level contribution between minimum and maximum values (Figure~\ref{fig:sensitivity}). This provides a rudimentary measure of sensitivity, which we can compare across parameters and timescales. At 2100, the largest sensitivity is to ice shelf basal melt, \gamma. 


•	The sensitivity of Antarctic SLC to parameters is a time-dependant question. Gamma starts our big very early on, but other parameters such as n and m catch up by 2300.
•	Glacial isostatic adjustment has negligible effect on SLC by 2100 and only a small effect by 2300.
•	Sensitivity to bed friction regularisation is negligible at 2100 and still small ~15cm by 2300.
•	Sensitivity to viscosity coefficient regularisation is the second largest of all parameters by 2100 ~10cm, second only to gamma. By 2300, the parameter also plays a role in the stability of Pine Island. However, the SLC sensitivity still becomes less significant relative to other parameters.







\section{Discussion}



\conclusions  %% \conclusions[modified heading if necessary]


%% The following commands are for the statements about the availability of data sets and/or software code corresponding to the manuscript.
%% It is strongly recommended to make use of these sections in case data sets and/or software code have been part of your research the article is based on.

\codeavailability{TEXT} %% use this section when having only software code available


\dataavailability{TEXT} %% use this section when having only data sets available


\codedataavailability{TEXT} %% use this section when having data sets and software code available


\sampleavailability{TEXT} %% use this section when having geoscientific samples available


\videosupplement{TEXT} %% use this section when having video supplements available


\appendix
\section{}    %% Appendix A

\subsection{}     %% Appendix A1, A2, etc.


\noappendix       %% use this to mark the end of the appendix section. Otherwise the figures might be numbered incorrectly (e.g. 10 instead of 1).

%% Regarding figures and tables in appendices, the following two options are possible depending on your general handling of figures and tables in the manuscript environment:

%% Option 1: If you sorted all figures and tables into the sections of the text, please also sort the appendix figures and appendix tables into the respective appendix sections.
%% They will be correctly named automatically.

%% Option 2: If you put all figures after the reference list, please insert appendix tables and figures after the normal tables and figures.
%% To rename them correctly to A1, A2, etc., please add the following commands in front of them:

\appendixfigures  %% needs to be added in front of appendix figures

\appendixtables   %% needs to be added in front of appendix tables

%% Please add \clearpage between each table and/or figure. Further guidelines on figures and tables can be found below.



\authorcontribution{TEXT} %% this section is mandatory

\competinginterests{TEXT} %% this section is mandatory even if you declare that no competing interests are present

\disclaimer{TEXT} %% optional section

\begin{acknowledgements}
TEXT
\end{acknowledgements}




%% REFERENCES

%% The reference list is compiled as follows:

\begin{thebibliography}{}

\bibitem[AUTHOR(YEAR)]{LABEL1}
REFERENCE 1

\bibitem[AUTHOR(YEAR)]{LABEL2}
REFERENCE 2

\end{thebibliography}

%% Since the Copernicus LaTeX package includes the BibTeX style file copernicus.bst,
%% authors experienced with BibTeX only have to include the following two lines:
%%
%% \bibliographystyle{copernicus}
%% \bibliography{example.bib}
%%
%% URLs and DOIs can be entered in your BibTeX file as:
%%
%% URL = {http://www.xyz.org/~jones/idx_g.htm}
%% DOI = {10.5194/xyz}


%% LITERATURE CITATIONS
%%
%% command                        & example result
%% \citet{jones90}|               & Jones et al. (1990)
%% \citep{jones90}|               & (Jones et al., 1990)
%% \citep{jones90,jones93}|       & (Jones et al., 1990, 1993)
%% \citep[p.~32]{jones90}|        & (Jones et al., 1990, p.~32)
%% \citep[e.g.,][]{jones90}|      & (e.g., Jones et al., 1990)
%% \citep[e.g.,][p.~32]{jones90}| & (e.g., Jones et al., 1990, p.~32)
%% \citeauthor{jones90}|          & Jones et al.
%% \citeyear{jones90}|            & 1990



%% FIGURES

%% When figures and tables are placed at the end of the MS (article in one-column style), please add \clearpage
%% between bibliography and first table and/or figure as well as between each table and/or figure.

% The figure files should be labelled correctly with Arabic numerals (e.g. fig01.jpg, fig02.png).


%% ONE-COLUMN FIGURES

%%f
%\begin{figure}[t]
%\includegraphics[width=8.3cm]{FILE NAME}
%\caption{TEXT}
%\end{figure}
%
%%% TWO-COLUMN FIGURES
%
%%f
%\begin{figure*}[t]
%\includegraphics[width=12cm]{FILE NAME}
%\caption{TEXT}
%\end{figure*}
%
%
%%% TABLES
%%%
%%% The different columns must be seperated with a & command and should
%%% end with \\ to identify the column brake.
%
%%% ONE-COLUMN TABLE
%
%%t
%\begin{table}[t]
%\caption{TEXT}
%\begin{tabular}{column = lcr}
%\tophline
%
%\middlehline
%
%\bottomhline
%\end{tabular}
%\belowtable{} % Table Footnotes
%\end{table}
%
%%% TWO-COLUMN TABLE
%
%%t
%\begin{table*}[t]
%\caption{TEXT}
%\begin{tabular}{column = lcr}
%\tophline
%
%\middlehline
%
%\bottomhline
%\end{tabular}
%\belowtable{} % Table Footnotes
%\end{table*}
%
%%% LANDSCAPE TABLE
%
%%t
%\begin{sidewaystable*}[t]
%\caption{TEXT}
%\begin{tabular}{column = lcr}
%\tophline
%
%\middlehline
%
%\bottomhline
%\end{tabular}
%\belowtable{} % Table Footnotes
%\end{sidewaystable*}
%
%
%%% MATHEMATICAL EXPRESSIONS
%
%%% All papers typeset by Copernicus Publications follow the math typesetting regulations
%%% given by the IUPAC Green Book (IUPAC: Quantities, Units and Symbols in Physical Chemistry,
%%% 2nd Edn., Blackwell Science, available at: http://old.iupac.org/publications/books/gbook/green_book_2ed.pdf, 1993).
%%%
%%% Physical quantities/variables are typeset in italic font (t for time, T for Temperature)
%%% Indices which are not defined are typeset in italic font (x, y, z, a, b, c)
%%% Items/objects which are defined are typeset in roman font (Car A, Car B)
%%% Descriptions/specifications which are defined by itself are typeset in roman font (abs, rel, ref, tot, net, ice)
%%% Abbreviations from 2 letters are typeset in roman font (RH, LAI)
%%% Vectors are identified in bold italic font using \vec{x}
%%% Matrices are identified in bold roman font
%%% Multiplication signs are typeset using the LaTeX commands \times (for vector products, grids, and exponential notations) or \cdot
%%% The character * should not be applied as mutliplication sign
%
%
%%% EQUATIONS
%
%%% Single-row equation
%
%\begin{equation}
%
%\end{equation}
%
%%% Multiline equation
%
%\begin{align}
%& 3 + 5 = 8\\
%& 3 + 5 = 8\\
%& 3 + 5 = 8
%\end{align}
%
%
%%% MATRICES
%
%\begin{matrix}
%x & y & z\\
%x & y & z\\
%x & y & z\\
%\end{matrix}
%
%
%%% ALGORITHM
%
%\begin{algorithm}
%\caption{...}
%\label{a1}
%\begin{algorithmic}
%...
%\end{algorithmic}
%\end{algorithm}
%
%
%%% CHEMICAL FORMULAS AND REACTIONS
%
%%% For formulas embedded in the text, please use \chem{}
%
%%% The reaction environment creates labels including the letter R, i.e. (R1), (R2), etc.
%
%\begin{reaction}
%%% \rightarrow should be used for normal (one-way) chemical reactions
%%% \rightleftharpoons should be used for equilibria
%%% \leftrightarrow should be used for resonance structures
%\end{reaction}
%
%
%%% PHYSICAL UNITS
%%%
%%% Please use \unit{} and apply the exponential notation (e.g. 20\,\unit{W\,m^{-2}})


\end{document}
